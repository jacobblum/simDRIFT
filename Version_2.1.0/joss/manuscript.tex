% Options for packages loaded elsewhere
\PassOptionsToPackage{unicode}{hyperref}
\PassOptionsToPackage{hyphens}{url}
%
\documentclass[
]{article}
\usepackage{amsmath,amssymb}
\usepackage{iftex}
\ifPDFTeX
  \usepackage[T1]{fontenc}
  \usepackage[utf8]{inputenc}
  \usepackage{textcomp} % provide euro and other symbols
\else % if luatex or xetex
  \usepackage{unicode-math} % this also loads fontspec
  \defaultfontfeatures{Scale=MatchLowercase}
  \defaultfontfeatures[\rmfamily]{Ligatures=TeX,Scale=1}
\fi
\usepackage{lmodern}
\ifPDFTeX\else
  % xetex/luatex font selection
\fi
% Use upquote if available, for straight quotes in verbatim environments
\IfFileExists{upquote.sty}{\usepackage{upquote}}{}
\IfFileExists{microtype.sty}{% use microtype if available
  \usepackage[]{microtype}
  \UseMicrotypeSet[protrusion]{basicmath} % disable protrusion for tt fonts
}{}
\makeatletter
\@ifundefined{KOMAClassName}{% if non-KOMA class
  \IfFileExists{parskip.sty}{%
    \usepackage{parskip}
  }{% else
    \setlength{\parindent}{0pt}
    \setlength{\parskip}{6pt plus 2pt minus 1pt}}
}{% if KOMA class
  \KOMAoptions{parskip=half}}
\makeatother
\usepackage{xcolor}
\setlength{\emergencystretch}{3em} % prevent overfull lines
\providecommand{\tightlist}{%
  \setlength{\itemsep}{0pt}\setlength{\parskip}{0pt}}
\setcounter{secnumdepth}{-\maxdimen} % remove section numbering
\ifLuaTeX
  \usepackage{selnolig}  % disable illegal ligatures
\fi
\IfFileExists{bookmark.sty}{\usepackage{bookmark}}{\usepackage{hyperref}}
\IfFileExists{xurl.sty}{\usepackage{xurl}}{} % add URL line breaks if available
\urlstyle{same}
\hypersetup{
  pdftitle={Simulated Diffusion in Realistic Imagaing Features of Tissue (Sim-DRIFT)},
  hidelinks,
  pdfcreator={LaTeX via pandoc}}

\title{Simulated Diffusion in Realistic Imagaing Features of Tissue
(Sim-DRIFT)}
\author{}
\date{01 June 2023}

\begin{document}
\maketitle

\hypertarget{summary}{%
\section{Summary}\label{summary}}

This library, \texttt{simDRIFT}, provides rapid and flexible Monte-Carlo
simulations of diffusion-weighted magnetic resonance imaging (dMRI),
which we expect to be useful for dMRI signal processing model
development and validation purposes. The primary focus of this library
is forward simulations of modular self-diffusion processes within an
ensemble of nuclear magnetic resonance (NMR) active nuclei (``spins'')
residing in complex, biophysical tissue systems. To acheive a large
variety of tissue configurations, \texttt{simDRIFT} provides support for
\(n\) fiber bundles (with user-defined radii, intrinsic diffusivities,
orientation angles, and densities) and \(m\) cells (with user-defined
radii and volume fractions). simDrift is written in Python (Python
Software Foundation, {[}@VanRossum2010{]}) and supported by a Numba
{[}@Lam2015{]} backend. Thus, \texttt{simDRIFT} benefits from Numba's
CUDA API, allowing the simulation of individual spin trajectories to be
performed in parallel on single Graphics Processing Unit (GPU) threads.
The resulting performance gains support \texttt{simDRIFT}'s aim to
provide a customizable tool for the rapid prototyping of diffusion
models, ground-truth model validation, and in silico phantom production.

\hypertarget{statement-of-need}{%
\section{Statement of need}\label{statement-of-need}}

Monte Carlo simulations are particularly effective at generating
synthetic diffusion MRI data from complex, biophysically-accurate
imaging voxels with known ground-truth microstructural parameters.
Consequently, such simulations of the Brownian self-diffusion process
have proven useful for developing and validating signal processing
models {[}@Chiang:2014; @Ye:2020{]}. Existing Monte Carlo simulators
typically rely on meshes to discretize the computational domain (see
e.g.,
{[}@Panagiotaki2010;@Yeh2013;@Ianus2016;@Kerkelae2020;@RafaelPatino2020{]}).
While this approach does allow for the representation of complex and
finely-detailed microstructural elements, creating meshes for the
biologically-relevant 3D geometries found in typical imaging voxels can
be difficult and may therefore present a barrier to wide use among
researchers who lack experience and training in computational
mathematics. The software encompassed by \texttt{simDRIFT} therefore
fulfills a presently-unmet need by allowing for mesh-free Monte Carlo
simulations of dMRI that unifies researcher's needs for computational
performance and biophysical realism with an easy-to-use and
highly-configurable software.

\texttt{simDRIFT} was designed to be used by researchers of all
disciplines and focuses who are working with diffusion MRI. Multiple
scientific publications which utilize this library are currently in
production. The wide customizability, high computational speed, and
massively-parallel design will provide avenues for improved model
development pipelines and thorough inter-model comparisons, among other
potential applications. These same traits may also make
\texttt{simDRIFT} useful for instructors of signal processing or
diffusion imaging courses.

\hypertarget{features}{%
\section{Features}\label{features}}

Given the coarse-graining of detailed microstructural features (fiber
bending, etc\ldots) observed at experimentally realistic diffusion times
and voxel sizes ({[}@Novikov2018{]}), \texttt{simDRIFT} represents
fibers as narrow cylinders, or ``sticks'', and cells as isotropic
spheres, or ``balls'' (Behrens et. al., 2003). The library allows users
to construct voxel geometries described by user-defined microstructural
and scanning parameters. Specifically, \texttt{simDRIFT} simulates the
diffusion MRI signal generated from the self-diffusion of water in an
isotropic imaging voxel of length \(L_{\mathrm{voxel}}\) that contains
\(n \in [0,4]\) distinct fiber bundles and \(m \in [0,2]\) distinct
cells types, according to the user's selection of diffusion imaging
protocol, diffusion time (\(\Delta\)), time-step size \(\mathrm{d}t\),
and desired free water diffusivity \(D_{FW}\). Within each simulated
voxel, the user also has control over the properties of each fiber/cell
type. For each fiber bundle, users define the desired orientation (via
the angle \(\theta\) formed between the bundle and the \(z\) axis),
intrinsic diffusivity (\(D_{i}\)), axonal radius (\(R_{i}\)), and voxel
volume fraction \(V_{i}/V_{\mathrm{vox}}\). For each cell type, users
similarly define the desired radius \(R_{j}\) and voxel volume fraction
\(V_{j}/V_{\mathrm{vox}}\). Examples of such voxel configurations can be
seen in {[}Figure 1{]}.

Figure 1: Example simulated spin trajectories from an imaging voxel
featuring two fiber bundles (red, blue) with various orientations (θ =
0°, 30°, 60°, 90°), along with cells (green)

For each time step \(\mathrm{d}t\) in the simulation, each tissue
compartment's resident spins are displaced along a randomly chosen
direction with a compartment-dependent distance
\(\mathrm{d}L = \sqrt{6D_{0}dt}\). This process is repeated until the
target diffusion time of \(\Delta\) is reached. For diffusion times
shorter than the expected pre-exchange lifetime of intracellular water,
it is safe to assume no exchange between tissue microstructures. The
inter-compartmental exchange of water is computationally forbidden via
within-timestep rejection of proposed moves beyond the boundaries of
each spin's domain.

\hypertarget{acknowledgments}{%
\section{Acknowledgments}\label{acknowledgments}}

The authors would like to acknowledge the early contributions of Chunyu
Song and Anthony Wu. We are also immensly grateful for the supportive
and clarifying discussions with Professor Sheng-Kwei Song, whose insight
helped to clarify the trajectory of this project.

\hypertarget{references}{%
\section{References}\label{references}}

\end{document}
